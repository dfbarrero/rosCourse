%%%%%%%%%%%%%%%%%%%%%%%%%%%%%%%%%%%%%%%%%%%%%%%%%%%%%%%%%%%%%%%%%
% MUW Presentation
% LaTeX Template
% Version 1.0 (27/12/2016)
%
% License:
% CC BY-NC-SA 4.0 (http://creativecommons.org/licenses/by-nc-sa/3.0/)
%
% Created by:
% Nicolas Ballarini, CeMSIIS, Medical University of Vienna
% nicoballarini@gmail.com
% http://statistics.msi.meduniwien.ac.at/
%
% Customized for UAH by:
% David F. Barrero, Departamento de Automática, UAH
%%%%%%%%%%%%%%%%%%%%%%%%%%%%%%%%%%%%%%%%%%%%%%%%%%%%%%%%%%%%%%%%%

\documentclass[10pt,compress]{beamer} % Change 10pt to make fonts of a different size
\mode<presentation>

\usepackage[spanish]{babel}
\usepackage{fontspec}
\usepackage{tikz}
\usepackage{etoolbox}
\usepackage{xcolor}
\usepackage{xstring}
\usepackage{listings}

\usetheme{UAH}
\usecolortheme{UAH}
\setbeamertemplate{navigation symbols}{} 
\setbeamertemplate{caption}[numbered]

%%%%%%%%%%%%%%%%%%%%%%%%%%%%%%%%%%%%%%%%%%%%%%%%%%%%%%%%%%%%%%%%%
%% Presentation Info
\title[Programming ROS with Python]{Programming ROS with Python}
\author{}
\institute{\asignatura}
\date{}
%%%%%%%%%%%%%%%%%%%%%%%%%%%%%%%%%%%%%%%%%%%%%%%%%%%%%%%%%%%%%%%%%


%%%%%%%%%%%%%%%%%%%%%%%%%%%%%%%%%%%%%%%%%%%%%%%%%%%%%%%%%%%%%%%%%
%% Descomentar para habilitar barra de navegación superior
\ponerNavegacion
%%%%%%%%%%%%%%%%%%%%%%%%%%%%%%%%%%%%%%%%%%%%%%%%%%%%%%%%%%%%%%%%%

%%%%%%%%%%%%%%%%%%%%%%%%%%%%%%%%%%%%%%%%%%%%%%%%%%%%%%%%%%%%%%%%%
%% Configuración de logotipos en portada
%% Opacidad de los logotipos
\newcommand{\opacidad}{1}
%% Descomentar para habilitar logotipo en pié de página de portada
\renewcommand{\logoUno}{Images/isg.png}
%% Descomentar para habilitar logotipo en pié de página de portada
%\renewcommand{\logoDos}{Images/CCLogo.png}
%% Descomentar para habilitar logotipo en pié de página de portada
%\renewcommand{\logoTres}{Images/ALogo.png}
%% Descomentar para habilitar logotipo en pié de página de portada
%\renewcommand{\logoCuatro}{Images/ELogo.png}
%%%%%%%%%%%%%%%%%%%%%%%%%%%%%%%%%%%%%%%%%%%%%%%%%%%%%%%%%%%%%%%%%

%%%%%%%%%%%%%%%%%%%%%%%%%%%%%%%%%%%%%%%%%%%%%%%%%%%%%%%%%%%%%%%%%
%% FOOTLINE
%% Comment/Uncomment the following blocks to modify the footline
%% content in the body slides. 


%% Option A: Title and institute
\footlineA
%% Option B: Author and institute
%\footlineB
%% Option C: Title, Author and institute
%\footlineC
%%%%%%%%%%%%%%%%%%%%%%%%%%%%%%%%%%%%%%%%%%%%%%%%%%%%%%%%%%%%%%%%%

\begin{document}

%%%%%%%%%%%%%%%%%%%%%%%%%%%%%%%%%%%%%%%%%%%%%%%%%%%%%%%%%%%%%%%%%
% Use this block for a blue title slide with modified footline
{\titlepageBlue
    \setbeamertemplate{headline}{}
	\setbeamercolor{frametitle}{bg=black}
	\setbeamercolor{normal text}{bg=black}
    \begin{frame}
        \titlepage
    \end{frame}
}

\begin{frame}[plain]{}
   \begin{block}{Objectives}
       \begin{itemize}
	   	\item Introduce the \texttt{catkin} build system
       	\item Implement ROS nodes with Python
       \end{itemize}
   \end{block}

   \begin{block}{Bibliography}
       Rospy package tutorials \href{http://wiki.ros.org/rospy/Tutorials}{(Link)}
   \end{block}

\end{frame}

{
\eliminarNavegacion
\begin{frame}[shrink]{Table of Contents}
 \frametitle{Table of Contents}
 \tableofcontents
  % You might wish to add the option [pausesections]
\end{frame}
}

\section{Overview}

\begin{frame}{Overview}
	Two native languages: C++ and Python
	\begin{itemize}
		\item C++ good for high performance
		\item Python good for prototyping
  	\end{itemize}
	ROS has its own build system, named \alert{catkin}
	\begin{itemize}
		\item Similar to \texttt{make}
		\item Built on \texttt{CMake}
		\item Make uses \texttt{Makefile}, catkin uses \texttt{CMakeLists.txt}
		\item[] \texttt{(Note: \texttt{rosbuild} is deprecated)}
  	\end{itemize}
	As almost everything in ROS, catkin is a package
	\begin{itemize}
		\item Documented in \url{http://wiki.ros.org/catkin}
		\item \href{http://wiki.ros.org/catkin/conceptual\_overview}{(Suggested read)}
	\end{itemize}
\end{frame}

\section{Catkin workspaces}
\subsection{Spaces}
\begin{frame}[fragile]{Catkin workspaces}{Spaces (I)}
	Catkin requires a \alert{workspace}
	\begin{itemize}
		\item Folder where install, modify and build packages
		\item Located in \texttt{\textasciitilde/catkin\_ws} in our VM
	\end{itemize}
	\vspace{0.6cm}
	Three folders (\alert{spaces}, in ROS' terminology) in a workspace
 	 	\begin{itemize}
		\item[\texttt{src}] Packages source code
		\item[\texttt{build}] Intermediate build files
		\item[\texttt{devel}] Intermediate installation files. \textit{Environment scripts}
		\end{itemize}
\end{frame}

\begin{frame}[fragile]{Catkin workspaces}{Spaces (II)}
\tiny{
    \begin{columns}
 	   \column{.80\textwidth}
	   \begin{block}{Typical catkin workspace}
\begin{verbatim}
workspace_folder/         -- WORKSPACE
  src/                    -- SOURCE SPACE
      CMakeLists.txt        -- The 'toplevel' CMake file
      package_1/
          CMakeLists.txt
          package.xml
          ...
      package_n/
          CMakeLists.txt
          package.xml
          ...
  build/                  -- BUILD SPACE
      CATKIN_IGNORE         -- Keeps catkin from walking this directory
  devel/                  -- DEVELOPMENT SPACE (set by CATKIN_DEVEL_PREFIX)
      bin/
      etc/
      include/
      lib/
      share/
      .catkin
      env.bash
      setup.bash
      setup.sh
      ...
\end{verbatim}
	   \end{block}
	   \end{columns}
	   }
\end{frame}

\subsection{Workspace creation}
\begin{frame}{Catkin workspaces}{Workspace creation}
	Creation of a new catkin workspace
	\begin{enumerate}
		\item Create folder: \texttt{mkdir -p \textasciitilde/catkin\_ws/src}
		\item Change working directory: \texttt{cd \textasciitilde/catkin\_ws/src}
		\item Initialize WS: \texttt{catkin\_init\_workspace}
  	\end{enumerate}

	(Already done in the VM)
\end{frame}

\section{Catkin packages}
\subsection{Package creation}
\begin{frame}[fragile]{Catkin packages}{Package creation}
	\vspace{-0.2cm}
    \begin{columns}
 	   \column{.40\textwidth}
	   \begin{block}{Simplest package}
\begin{verbatim}
myPackage/
    CMakeLists.txt
    package.xml
\end{verbatim}
	   \end{block}
	\end{columns}
	\bigskip
	Creation of a new package (from the workspace)
	\begin{enumerate}
		\item \texttt{catkin\_create\_pkg <package\_name> [depend1] [depend2] [depend3]}
		\item[] Example: \texttt{catkin\_create\_pkg myPackage std\_msgs rospy}
		\item Customize \texttt{package.xml}
	\end{enumerate}
	Warning: Package still not available
\end{frame}

\subsection{Package build}
\begin{frame}[fragile]{Catkin packages}{Package build}
	To build the workspace: \texttt{catkin\_make targets}
	\begin{itemize}
		\item From the folder \texttt{src}
		\item Uses a \texttt{CMakeLists.txt} file
		\item By default builds all the packages
	\end{itemize}
	To execute the node
	\begin{enumerate}
		\item Make it accesible: \texttt{source devel/setup.bash}
		\item Execute it: \texttt{rosrun package node}
		\item[] Warning: package is given in \texttt{packages.xml}
	\end{enumerate}
	To install a package: \texttt{catkin\_make install}
\end{frame}

\subsection{Exercise}
\begin{frame}[fragile]{Catkin packages}{Exercise (I)}
	Implement a ``Hello, world'' node
	\begin{itemize}
		\item Create a \texttt{exercises} package dependent on \texttt{rospy}
		\item Customize \texttt{package.xml}
		\item Create a folder named \texttt{scripts}
		\item Create a file named \texttt{hello.py}
		\item Give execution permissions to \texttt{hello.py}
		\item Edit the file (next slide)
		\item Build the project
		\item Run \texttt{source devel/setup.bash}
		\item Execute \texttt{roscore}
		\item In other tab, execute the node (\texttt{hello.py})
	\end{itemize}
\end{frame}

\begin{frame}[fragile]{Catkin packages}{Exercise (II)}
	\begin{block}{scripts/hello.py}
        \lstinputlisting[language=python, basicstyle=\ttfamily\scriptsize]{code/hello.py}
	\end{block}
	Extra points: Run \texttt{hello.py} with a launch file
\end{frame}

\section{Nodes programming with Python}

\subsection{Topics}
\begin{frame}[fragile]{Nodes programming with Python}{Topics (I)}
	\begin{itemize}
		\item Code stored in folder \texttt{scripts}
		\item Scripts must have execution permissions
		\item Must import \texttt{rospy} Python module
		\item Must init the node: \texttt{rospy.init\_node('name')}
		\item Convenient assets
			\begin{itemize}
			\item \texttt{Rate} class and \texttt{sleep()} method
			\item \texttt{rospy.is\_shutdown()}
			\end{itemize}
	\end{itemize}
\end{frame}

\begin{frame}{Nodes programming with Python}{Topics (II)}
    \vspace{-0.2cm}
	\begin{block}{scripts/talker.py}
    \vspace{-0.2cm}
        \lstinputlisting[language=python, basicstyle=\ttfamily\scriptsize]{code/talker.py}
    \vspace{-0.2cm}
	\end{block}
\end{frame}

\begin{frame}[fragile]{Nodes programming with Python}{Topics (III)}
	\label{listener}
    \vspace{-0.4cm}
	\begin{block}{scripts/listener.py}
    \vspace{-0.2cm}
        \lstinputlisting[language=python, basicstyle=\ttfamily\scriptsize]{code/listener.py}
    \vspace{-0.2cm}
	\end{block}

    \vspace{-1.7cm}
	\begin{columns}
 	\column{.4\textwidth}
 	\column{.4\textwidth}
    \vspace{-0.2cm}
	\begin{exampleblock}{std\_msgs/msg/String.msg}
\begin{verbatim}
string data
\end{verbatim}
    \vspace{-0.3cm}
	\end{exampleblock}
	\end{columns}
\end{frame}

\begin{frame}{Nodes programming with Python}{Topics (IV): Exercise}
	Run the example
	\begin{enumerate}
		\item Make them accesible: \texttt{source devel/setup.bash}
		\item Initialize ROS: \texttt{roscore}
		\item Execute the nodes: 
			\begin{itemize}
				\item \texttt{rosrun exercises talker.py}
				\item \texttt{rosrun exercises listener.py}
			\end{itemize}
	\end{enumerate}
\end{frame}

\begin{frame}{Nodes programming with Python}{Topics (V)}
	ROS only provides a callback to read topics
	\begin{itemize}
		\item Reading just the last message is not out-of-the-box
	\end{itemize}
	A common practice is to have a listener in background
	\begin{itemize}
		\item Updates a global variable with the message
	\end{itemize}

    \vspace{-0.2cm}
	\begin{exampleblock}{Example}
    \vspace{-0.2cm}
        \lstinputlisting[language=python, basicstyle=\ttfamily\scriptsize]{code/odom.py}
    \vspace{-0.3cm}
	\end{exampleblock}

\end{frame}

\begin{frame}{Nodes programming with Python}{Topics (VI): Exercise}
	Excercise:
	\begin{enumerate}
		\item Modify listener in slide~\pageref{listener} to store the last message
		\item Show the message five times per second
		\begin{itemize}
			\item Hint: Use \texttt{rospy.Rate()} and \texttt{rospy.spin()}
		\end{itemize}
	\end{enumerate}
\end{frame}

\subsection{Messages}
\begin{frame}[fragile]{Nodes programming with Python}{Messages (I)}
	Same structure in Python than in the \texttt{msg} file
    \begin{columns}
 	   \column{.50\textwidth}
	\begin{exampleblock}{\texttt{geometry\_msgs/Twist.msg}}
	\begin{verbatim}
Vector3  linear
Vector3  angular
\end{verbatim}
	\end{exampleblock}
	\begin{exampleblock}{\texttt{geometry\_msgs/Vector3.msg}}
	\begin{verbatim}
float64 x
float64 y
float64 z
\end{verbatim}
	\end{exampleblock}
 	   \column{.50\textwidth}
    \vspace{-0.2cm}
	\begin{block}{Message usage}
    \vspace{-0.2cm}
        \lstinputlisting[language=python, basicstyle=\ttfamily\scriptsize]{code/mensaje.py}
    \vspace{-0.2cm}
	\end{block}
	\end{columns}
\end{frame}

\begin{frame}[fragile]{Nodes programming with Python}{Messages (II)}
	\vspace{-0.8cm}
	\begin{center}
	\begin{tabular}{lc} \hline
	  {\sc Type} & {\sc Keyword}\\ \hline
	Integer & int8, int16, int32, int64 (plus uint*)\\
	Float & float32, float64\\
	String & string\\
	Time & time, duration\\
	Variable-length array &  array[] (example: float32[])\\
	Fixed-length array	 & array[C] (example: float32[5])\\
	Struct & other msg files\\
	  \hline  
	\end{tabular}
	\end{center}
	\vspace{-0.2cm}
	Custom messages need wrappers classes
	\begin{itemize}
		\item Automatically generated by catkin
		\item Requires configure dependencies (i.e. set up \texttt{packages.xml} and \texttt{CMakeLists.txt})
	\end{itemize}
\end{frame}

\begin{frame}[fragile]{Nodes programming with Python}{Messages (III)}
	Implement the following tasks:
	\begin{enumerate}
		\item Execute roscore
		\item Execute \texttt{rosrun turtlesim turtlesim\_node}
		\item Implement a node that moves the turtle forward with constant velocity
		\item Implement a node that shows the turtle pose
	\end{enumerate}
	Execute \texttt{roslaunch stdr\_launchers server\_with\_map\_and\_gui\_plus\_robot.launch} and write a node that
	\begin{enumerate}
		\item Shows the odometry as it appears
		\item Shows sonar measures as they appear
		\item Stores the last odometry and sensor measures
	\end{enumerate}

\end{frame}

\subsection{Services}
\begin{frame}[fragile]{Nodes programming with Python}{Services: Setting up the build-system (I)}
	Automatic generation of proxies (proxy $=$ interface)
	\begin{itemize}
		\item Python and C++
		\item Similar messages and services
		\item Stored in \small{\texttt{\$(WS)/devel/lib/python2.7/dist-packages}}
	\end{itemize}
	Modify \texttt{packages.xml} and \texttt{CMakeLists.txt} to inform catkin
	\begin{itemize}
		\item \href{http://wiki.ros.org/ROS/Tutorials/CreatingMsgAndSrv}{(More info)} \href{http://answers.ros.org/question/114806/tutorial-116-importerror-no-module-named-beginner\_tutorialssrv-with-catkin-system-build/}{(More)}
	\end{itemize}
	Service creation process:
	\begin{enumerate}
			\item Create the \texttt{srv} file in folder \texttt{srv}
			\item Enable code generation by editting \texttt{packages.xml}
			\begin{verbatim}
<build_depend>message_generation</build_depend>
<run_depend>message_runtime</run_depend>
\end{verbatim}
	\end{enumerate}
\end{frame}

\begin{frame}[fragile]{Nodes programming with Python}{Services: Setting up the build-system (II)}
	\begin{enumerate}
		\setcounter{enumi}{2}
		\item Add dependency to \texttt{CMakeLists.txt}, uncommenting 
			\begin{verbatim}
find_package(catkin REQUIRED COMPONENTS
  roscpp
  rospy
  std_msgs
  message_generation
}
\end{verbatim}
		\item Add service file
			\begin{verbatim}
add_service_files(
  FILES
  AddTwoInts.srv
)
\end{verbatim}
		\end{enumerate}
\end{frame}

\begin{frame}[fragile]{Nodes programming with Python}{Services: Service provider (I)}
	Two components
	\begin{itemize}
		\item Provider and consumer (or server and client)
		\item Both uses proxyes (like local function calls)
		\item Both implemented in nodes
	\end{itemize}
	Service provider
	\begin{itemize}
		\item Method \texttt{rospy.Service()}
		\item Request as \texttt{fooRequest}
		\item Response as \texttt{fooResponse}
	\end{itemize}

	\vspace{-2cm}

    \begin{columns}
 	   \column{.70\textwidth}
 	   \column{.30\textwidth}
	\begin{exampleblock}{AddTwoInts.srv}
		\begin{verbatim}
int64 a
int64 b
---
int64 sum
\end{verbatim}
    \vspace{-0.3cm}
	\end{exampleblock}
	\end{columns}

\end{frame}

\begin{frame}{Nodes programming with Python}{Services: Service provider (II)}
    \vspace{-0.2cm}
	\begin{block}{scripts/add\_two\_ints\_server.py}
    \vspace{-0.2cm}
        \lstinputlisting[language=python, basicstyle=\ttfamily\scriptsize]{code/server.py}
    \vspace{-0.2cm}
	\end{block}
\end{frame}

\begin{frame}{Nodes programming with Python. Services: Service consumer (I)}
	Two methods
	\begin{itemize}
		\item Wait until service available: \texttt{rospy.wait\_for\_service()}
		\item Get proxy: \texttt{rospy.ServiceProxy()}
	\end{itemize}
	Exception: \texttt{rospy.ServiceException}
\end{frame}

\begin{frame}[plain]{Nodes programming with Python. Services: Service consumer (II)}
\tiny{
    \vspace{-0.2cm}
	\begin{block}{scripts/add\_two\_ints\_client.py.py}
    \vspace{-0.2cm}
        \lstinputlisting[language=python, basicstyle=\ttfamily\scriptsize]{code/client.py}
    \vspace{-0.2cm}
	\end{block}

	}
\end{frame}

\begin{frame}{Nodes programming with Python}{Services: Exercises}
	Run the previous examples
	\begin{enumerate}
		\item Make them accesible: \texttt{source dev/setup.bash}
		\item Initialize ROS: \texttt{roscore}
		\item Execute the nodes: 
			\begin{itemize}
				\item \texttt{rosrun myPackage add\_two\_ints\_client.py}
				\item \texttt{rosrun myPackage add\_two\_ints\_client.py 5 6}
			\end{itemize}
	\end{enumerate}
	Execute STDR with a robot
	\begin{enumerate}
		\item Invoke programmatically a service to move the robot to coordinates $(15, 15)$
	\end{enumerate}
\end{frame}

\begin{frame}{Exercises}
	Implement the following tasks in ROS
	\begin{enumerate}
		\item Launch STDR with any robot
		\item Move the robot four distance units to the east
		\item[] Hint: Use odometry
		\item Move the robot four distance units to the east and then one to the north
		\item Move the robot to the opposite side of the map
	\end{enumerate}
\end{frame}

\end{document}
